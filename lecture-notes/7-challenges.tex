\documentclass[aspectratio=1610]{beamer}

\usepackage{amsmath}
\usepackage{multirow}
\usepackage{url}
\usepackage{hyperref}

\hypersetup{
colorlinks=false,
}

\usepackage{listings,calc,graphicx}

\title % [short title] (optional, use only with long paper titles)
{CPSC 1000: Introduction to Computer Science -- PART 2}

\subtitle{The project challenges} % (optional)

\author{Robert Benkoczi, C556\\\url{robert.benkoczi@uleth.ca}}
\date{25-oct-2018\\(Week 7)}

% for figures created with IPE
%\pdfpagebox5

\lstloadlanguages{C}
\lstset{language=C,tabsize=2,aboveskip=-22pt,belowskip=-22pt,keepspaces,
  basicstyle=\small\ttfamily,}


\begin{document}

\begin{frame}[plain]
\titlepage
\end{frame}

%%%%%%

\begin{frame}[t,plain]{Objectives}
\begin{itemize}
\item CPSC 1000 course project: students will program a three wheel
  robot controlled with an Arduino microcontroller that will seek a
  target object.
\item  Resources: 
\begin{itemize}
\item Arduino Notebook 
\item Arduino Programming Language
  Reference
\item The Adafruit motor shield documentation: 
\url{https://learn.adafruit.com/adafruit-motor-shield-v2-for-arduino/library-reference}
\item The course notes (the probing/action programming template,
  working with various sensors, etc.)
\end{itemize}
\end{itemize}
\end{frame}

%%%%%%%

\begin{frame}[t,plain,fragile]{Introduction}
The robot:

\begin{itemize}
\item Two motors controlling one rubber wheel each.
\item To move forward, choose the ``same'' speed to the two motors.
\item To turn, set different speeds to the two motors.
\item Various sensors can be attached to the body of the robot.
\end{itemize}

\end{frame}

%%%%%%%%%%

\begin{frame}[t,plain]{Three challenges (target seeking robot)}
The project is decomposed into three challenges, increasing in
complexity.

\medskip
\begin{itemize}
\item Task for all challenges: the robot should approach the target
  object (a box) within 20 cm from it, but without touching it.

\medskip
\item Setup for Challenge 1: robot facing the target object at
  distance 1 m from it.
 \item Setup for Challenge 2: robot facing the target object at an
   arbitrary distance between 1 m and 2 m.
\item Setup for Challenge 3: set at an arbitrary distance between 1 m
  and 2 m from the target, oriented arbitrarily.

\medskip
\item The target object: box with a source of light on top. 
\end{itemize}

\end{frame}


\end{document}
