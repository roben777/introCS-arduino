\documentclass[aspectratio=1610]{beamer}

\usepackage{amsmath}
\usepackage{multirow}
\usepackage{url}
\usepackage{hyperref}

\hypersetup{
colorlinks=false,
}

\usepackage{listings,calc,graphicx}

\title % [short title] (optional, use only with long paper titles)
{CPSC 1000: Introduction to Computer Science}

\subtitle{Pulse width modulation (PWM) with Arduino} % (optional)

\author{Robert Benkoczi, C556\\\url{robert.benkoczi@uleth.ca}}
\date{16-Oct-2018\\(Week 6)}

% for figures created with IPE
%\pdfpagebox5

\lstloadlanguages{C}
\lstset{language=C,tabsize=2,aboveskip=-22pt,belowskip=-22pt,keepspaces,
  basicstyle=\small\ttfamily,}


\begin{document}

\begin{frame}[plain]
\titlepage
\end{frame}

%%%%%%

\begin{frame}[t,plain]{Objectives}
\begin{itemize}
\item Students will write Arduino
  programs that output a digital signal simulating an analog signal,
  using pulse width modulation on the Arduino PWM digital pins.
\item  Resources: 
\begin{itemize}
\item Arduino Notebook 
\item Arduino Programming Language
  Reference
\end{itemize}
\end{itemize}
\end{frame}

%%%%%%%

\begin{frame}[plain,t]{Why analog signal output?}

\begin{itemize}
\item<1> Power source with adjustable voltage. DC motor speed =
  determined by the amount of power transferred from the electrical
  source (Ohm's law).

\item<2> Set the colour of an RGB LED.

\end{itemize}
\end{frame}


%%%%%%%

\begin{frame}[plain,t]{Pulse width modulation}
Arduino PWM frequency approx 490 Hz
\end{frame}

%%%%%%%

\begin{frame}[plain,t]{Programming}
\begin{itemize}
\item No setup needed.

\item \lstinline$analogWrite(pin, value)$:
\end{itemize}
\end{frame}

%%%%%%%

\begin{frame}[plain,t]{Question}
Can we implement PWM using digital output?
\end{frame}

%%%%%%%

\begin{frame}[plain,t]{Example}

Choose the LED intensity  randomly, and maintain it for 2 sec before
choosing another intensity.
\end{frame}


%%%%%%%

\begin{frame}[plain,t]{Example 2}
Set the PWM value proportional to the value of $x$ relative to a range
$m \le x \le M$.
\end{frame}

%%%%%%


\begin{frame}[plain]{PART II}
{\large Controlling motors using the Adafruit motor shield v. 2.3}
\end{frame}


%%%%%%

\begin{frame}[t,plain]{Objectives}
\begin{itemize}
\item Students will write Arduino
  programs that will control the speed of a simple 6V DC motor using
  an additional circuit: the Adafruit motor shield.
\item  Resources: 
\begin{itemize}
\item Arduino Notebook 
\item Arduino Programming Language
  Reference
\item The Adafruit motor shield documentation: 
\url{https://learn.adafruit.com/adafruit-motor-shield-v2-for-arduino/library-reference}
\end{itemize}
\end{itemize}
\end{frame}

%%%%%%%

\begin{frame}[t,plain,fragile]{Using the library:}

To load the library definitions, types, and functions:

\begin{semiverbatim}
\begin{lstlisting}
#include <Wire.h>
#include <Adafruit_MotorShield.h>
#include "utility/Adafruit_MS_PWMServoDriver.h"
\end{lstlisting}
\end{semiverbatim}


\medskip

To define the objects for the shield and motor(s):

\begin{semiverbatim}
\begin{lstlisting}
Adafruit_MotorShield AFMS = Adafruit_MotorShield(); 
Adafruit_DCMotor *myMotor = AFMS.getMotor(4);
\end{lstlisting}
\end{semiverbatim}
\end{frame}



\end{document}
