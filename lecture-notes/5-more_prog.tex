\documentclass[aspectratio=1610]{beamer}

\usepackage{amsmath}
\usepackage{multirow}
\usepackage{url}
\usepackage{hyperref}

\hypersetup{
colorlinks=false,
}

\usepackage{listings,calc,graphicx}

\title % [short title] (optional, use only with long paper titles)
{CPSC 1000: Introduction to Computer Science}

\subtitle{Arduino basic program template} % (optional)

\author{Robert Benkoczi, C556\\\url{robert.benkoczi@uleth.ca}}
\date{(9,11)-Oct-2018\\(Week 5)}

% for figures created with IPE
%\pdfpagebox5

\lstloadlanguages{C}
\lstset{language=C,tabsize=2,aboveskip=-22pt,belowskip=-22pt,keepspaces,
  basicstyle=\small\ttfamily,}


\begin{document}

\begin{frame}[plain]
\titlepage
\end{frame}

%%%%%%

\begin{frame}[t,plain]{Objectives}
\begin{itemize}
\item Using the Arduino Notebook and the Arduino Programming Language
  Reference, both available on Moodle, and the functions explained
  here, students will write Arduino programs that read input from
  sensors or the serial monitor, and perform actions based on these
  inputs. 
\item Students will write Arduino programs that receive data to/from
  the serial monitor.
\item Students will use \lstinline$random$ and \lstinline$randomSeed$
  functions to add ``chance'' to their programs.
\item Students will write loop statements.
\end{itemize}
\end{frame}

%%%%%%%
\begin{frame}[plain,t]{The general template (see Lab Assignment 3 and
    the bunus question)}


\end{frame}

%%%%%%%

\begin{frame}[plain,t]{Sending data to the serial monitor (recap)}

\begin{itemize}
\item \lstinline$Serial.begin(9600);$

\bigskip
\item \lstinline$Serial.println(``Hello world'');$

\smallskip
Example 1: write a ``hello world'' program which sends the message
to the serial monitor every 500 ms.

\bigskip
\bigskip
\item Chance: \lstinline$random(min, max)$

\smallskip
Example 2: display on serial monitor a random integer between 1 and
100, every 500 ms.

\bigskip
\item The need for \lstinline$randomSeed(seed)$. We can supply a different
  \emph{seed} value to the random number generator by measuring an
  unconnected analog pin (we measure noise).
\end{itemize}
\end{frame}


%%%%%%%

\begin{frame}[plain,t]{Receiving data from serial monitor}

Why? We can configure our project in non-trivial ways. 

\begin{itemize}
\item \lstinline$Serial.read()$: returns -1 if no data available, otherwise
returns the first byte read (byte = 8 bit integer).

\smallskip
Example 3: read one byte and echo it back to the serial monitor. 

\vspace*{.2\textheight}
\item Read the byte into a char variable. 

\end{itemize}
\end{frame}

%%%%%%%

\begin{frame}[plain,t]{Output only when character received: if
    statement}


\vfill
Example 4: \lstinline$Serial.available()$ - output only when a
character is received.

\end{frame}



%%%%%%%

\begin{frame}[plain,t]{Read a string (text value) from serial monitor}

String: sequence of characters. Can be stored into a \emph{String}
variable.

\bigskip
\begin{itemize}
\item Append characters to a String variable.

\smallskip
Example 5: read data as a string from serial monitor and echo it back.

\smallskip
\item while loop syntax:
\end{itemize}
\end{frame}


%%%%%%%

\begin{frame}[plain,t]{Read a numeric value from serial monitor}

\begin{itemize}
\item \lstinline$str.toInt()$ (\emph{str} is a String variable): Read
  a string, then convert it to an int.

\smallskip
Example 6: Define a \emph{readInt()} function. Use it to seed the
random number generator instead of using the value of noise.
\end{itemize}

\end{frame}


%%%%%%%

\begin{frame}[plain,t]{Exercises}

\begin{itemize}
\item Blink a LED, but make the LED skip every fourth blink.
\item Make the LED be on and off for a period of time
  that changes randomly.
\item Connect 10 LEDs to digital pins 2-11. Light one LED at a time,
  simulating a ``left to right'' motion. Hint: use a for loop to
  control the on/off pattern. 
\end{itemize}
\end{frame}


\end{document}
